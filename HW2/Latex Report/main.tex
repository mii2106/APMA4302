\documentclass[12pt]{article}

\usepackage[utf8x]{inputenc}
\usepackage[english]{babel}
\usepackage{subcaption}
\usepackage{amssymb,amsmath,amsthm,amsfonts}
\usepackage{calc}
\usepackage{graphicx}
\usepackage[shortlabels]{enumitem}

\usepackage{gensymb}
\usepackage{float}
\usepackage{natbib}
\usepackage{url}
\usepackage[utf8x]{inputenc}
\usepackage{amsmath}
\usepackage{graphicx}
\graphicspath{{images/}}
\usepackage{parskip}
\usepackage{fancyhdr}
\usepackage{vmargin}
\usepackage{mathrsfs}
\usepackage{graphicx}
\usepackage{subcaption}
\usepackage[dvipsnames]{xcolor}
\usepackage{longtable}
\usepackage{multicol}
\usepackage{multirow}
\usepackage{booktabs}

%tikzpicture
\usepackage{tikz}
\usepackage{scalerel}
\usepackage{pict2e}
\usepackage{tkz-euclide}
\usetikzlibrary{calc}
\usetikzlibrary{patterns,arrows.meta}
\usetikzlibrary{shadows}
\usetikzlibrary{external}
\usepackage{listings}
\usepackage{xcolor}

\definecolor{lightgray}{rgb}{0.95,0.95,0.95}  % gris muy claro
% más oscuro:
% \definecolor{lightgray}{rgb}{0.9,0.9,0.9}
% aún más plomo:
% \definecolor{lightgray}{rgb}{0.85,0.85,0.85}

\lstset{
  language=Python,
  basicstyle=\ttfamily\small,
  backgroundcolor=\color{lightgray},
  frame=single,
  rulecolor=\color{black},
  keywordstyle=\color{blue},
  stringstyle=\color{red},
  commentstyle=\color{green!50!black},
  showstringspaces=false,
  breaklines=true
}
%pgfplots
\usepackage{pgfplots}
\pgfplotsset{compat=newest}
\usepgfplotslibrary{statistics}
\usepgfplotslibrary{fillbetween}

\setmarginsrb{3 cm}{2.5 cm}{3 cm}{2.5 cm}{1 cm}{1.5 cm}{1 cm}{1.5 cm}

\title{Homework 2}					% Titulo
\author{Matías I. Inostroza (mii2106)}					% Autor
\date{\today}						% Fecha


\makeatletter
\let\thetitle\@title
\let\theauthor\@author
\let\thedate\@date
\makeatother

\pagestyle{fancy}
\fancyhf{}
\rhead{\theauthor}
\lhead{\thetitle}
\cfoot{\thepage}

\begin{document}

%%%%%%%%%%%%%%%%%%%%%%%%%%%%%%%%%%%%%%%%%%%%%%%%%%%%%%%%%%%%%%%%%%%%%%%%%%%%%%%%%%%%%%%%%

\begin{titlepage}
	\centering
    \vspace*{0.0 cm}
    \includegraphics[scale = 0.35]{Columbia_logo.png}\\[2.0 cm]	% Logo Universidad
    \textsc{\LARGE Columbia University}\\[2.0 cm]	% Nombre Universidad
	\textsc{\Large APMAE4302}\\[0.5 cm]				% Codigo Curso
	\textsc{\large Methods in computational science}\\[0.5 cm]		% Nombre Curso
	\rule{\linewidth}{0.2 mm} \\[0.4 cm]
	{ \huge \bfseries \thetitle}\\
	\rule{\linewidth}{0.2 mm} \\[1.5 cm]
	
	\begin{minipage}{0.4\textwidth}
		\begin{center} \large
			\emph{Author:}\\
			\theauthor\linebreak
			\end{center}
	\end{minipage}\\[2 cm]
	
	{February 23, 2026}\\[2 cm]
 
	\vfill
	
\end{titlepage}

%%%%%%%%%%%%%%%%%%%%%%%%%%%%%%%%%%%%%%%%%%%%%%%%%%%%%%%%%%%%%%%%%%%%%%%%%%%%%%%%%%%%%%%%%

\tableofcontents
\pagebreak

%%%%%%%%%%%%%%%%%%%%%%%%%%%%%%%%%%%%%%%%%%%%%%%%%%%%%%%%%%%%%%%%%%%%%%%%%%%%%%%%%%%%%%%%%
\section{Problem 1}

\textbf{1)} First, let’s assume that there are two problems described by:

\begin{equation}
    \mathbf{A}\mathbf{u} = \mathbf{b} \qquad \text{and} \qquad \mathbf{A}\hat{\mathbf{u}} = \hat{\mathbf{b}}
\end{equation}

By subtracting both:

\begin{eqnarray}
 \mathbf{A}\mathbf{u} -   \mathbf{A}\hat{\mathbf{u}} =  \mathbf{b}-\hat{\mathbf{b}}\\
  \mathbf{A}(\mathbf{u}-\hat{\mathbf{u}}) = \mathbf{b}-\hat{\mathbf{b}} \\
  \rightarrow  (\mathbf{u}-\hat{\mathbf{u}}) = \mathbf{A}^{-1}(\mathbf{b}-\hat{\mathbf{b}})
\end{eqnarray}

Considering $||\mathbf{A}\mathbf{x}||\le||\mathbf{A}||||\mathbf{x}||$

\begin{eqnarray}
    ||\mathbf{A}(\mathbf{u}-\hat{\mathbf{u}})|| \le \lVert\mathbf{A}\lVert\lVert(\mathbf{u}-\hat{\mathbf{u}})\lVert = \lVert\mathbf{A}\lVert\lVert\mathbf{A}^{-1}(\mathbf{b}-\hat{\mathbf{b}})\lVert \le \lVert\mathbf{A}\lVert\lVert\mathbf{A}^{-1}\lVert\lVert(\mathbf{b}-\hat{\mathbf{b}})\lVert 
\end{eqnarray}

Thus:

\begin{equation}
   \lVert\mathbf{A}\lVert\lVert(\mathbf{u}-\hat{\mathbf{u}})\lVert \le  \lVert\mathbf{A}\lVert\lVert\mathbf{A}^{-1}\lVert\lVert(\mathbf{b}-\hat{\mathbf{b}})\lVert  = \kappa (\mathbf{A})\lVert(\mathbf{b}-\hat{\mathbf{b}})\lVert 
\end{equation}

Also, it can be established that$\lVert \mathbf{A}\mathbf{u}\lVert = \lVert\mathbf{b}\lVert \le \lVert \mathbf{A}\lVert\lVert\mathbf{u}\lVert$, and therefore $\frac{\lVert\mathbf{b}\lVert}{\lVert\mathbf{u}\lVert} \le \lVert \mathbf{A}\lVert$

By multiplying both sides by $\lVert(\mathbf{u}-\hat{\mathbf{u}})\lVert$

\begin{equation}
 \frac{\lVert\mathbf{b}\lVert}{\lVert\mathbf{u}\lVert}   \lVert(\mathbf{u}-\hat{\mathbf{u}})\lVert \le   \lVert\mathbf{A}\lVert\lVert(\mathbf{u}-\hat{\mathbf{u}})\lVert
\end{equation}

Therefore, using Equations (6) and (7):

\begin{eqnarray}
   \frac{\lVert\mathbf{b}\lVert}{\lVert\mathbf{u}\lVert}   \lVert(\mathbf{u}-\hat{\mathbf{u}})\lVert \le   \kappa (\mathbf{A})\lVert(\mathbf{b}-\hat{\mathbf{b}})\lVert   \\
   \rightarrow    \frac{\lVert\mathbf{u}-\hat{\mathbf{u}}\lVert}{\lVert\mathbf{u}\lVert}    \le   \kappa (\mathbf{A})\frac{\lVert(\mathbf{b}-\hat{\mathbf{b}})\lVert }{\lVert \mathbf{b} \lVert}
\end{eqnarray}

\textbf{2)} 

From equation 3:
\begin{equation}
    ||\mathbf{A}(\mathbf{u}-\hat{\mathbf{u}})|| \le \lVert\mathbf{A}\lVert\lVert(\mathbf{u}-\hat{\mathbf{u}})\lVert \rightarrow ||\mathbf{r}|| \le \lVert\mathbf{A}\lVert\lVert \mathbf{e} \lVert
\end{equation}

Also, as $\mathbf{u} = \mathbf{A}^{-1}\mathbf{b}$, then:

\begin{equation}
    \lVert\mathbf{u}\lVert = \lVert \mathbf{A}^{-1}\mathbf{b}\lVert \le \lVert \mathbf{A}^{-1}\lVert \lVert \mathbf{b}\lVert 
\end{equation}

Multiplying Equation (10) by Equation (11):

\begin{eqnarray}
        ||\mathbf{r}||\lVert\mathbf{u}\lVert \le \lVert \mathbf{A}^{-1}\lVert \lVert\mathbf{A}\lVert \lVert \mathbf{b}\lVert\lVert \mathbf{e} \lVert\\
        ||\mathbf{r}||\lVert\mathbf{u}\lVert \le \kappa(\mathbf{A}) \lVert \mathbf{b}\lVert\lVert \mathbf{e} \lVert\\
        \frac{1}{\kappa(\mathbf{A})}\frac{\lVert \mathbf{r}\lVert}{\lVert \mathbf{b}\lVert} \le \frac{\lVert \mathbf{e}\lVert}{\lVert \mathbf{u}\lVert}
\end{eqnarray}

Since $\lVert \mathbf{e} \lVert = \lVert \mathbf{u}-\hat{\mathbf{u}} \lVert $ it follows that:

\begin{equation}
         \frac{1}{\kappa(\mathbf{A})}\frac{\lVert \mathbf{r}\lVert}{\lVert \mathbf{b}\lVert} \le \frac{\lVert \mathbf{e}\lVert}{\lVert \mathbf{u}\lVert} \le \kappa (\mathbf{A})\frac{\lVert \mathbf{r}\lVert }{\lVert \mathbf{b} \lVert}
\end{equation}

\section{Problem 2}
If $\mathbf{v}_j = \sin(j\pi x_i)$ is an eigenvector of $\mathbf{A}$, then it satisfies
$\mathbf{A}\mathbf{v}_j = \lambda_j \mathbf{v}_j$.

\begin{eqnarray}
(\mathbf{A}\mathbf{v}_j)_i &=& \frac{1}{h^2}(-v^{i-1} + 2v^i - v^{i+1}) \\
 &=& \frac{1}{h^2}(-\sin(j\pi (i-1)h ) + 2\sin(j\pi (i)h) - \sin(j\pi (i+1)h)
\end{eqnarray}

Define the angles $\theta = j\pi i h$ and $\phi = j\pi h$.

\begin{eqnarray}
(\mathbf{A}\mathbf{v}_j)_i 
\notag  &=& \frac{1}{h^2}(-\sin(\theta -\phi) + 2\sin(\theta) - \sin(\theta +\phi))\\
 \notag  &=&\frac{1}{h^2}(-\sin(\theta)\cos(\phi) + \sin(\phi)\cos(\theta) + 2\sin(\theta) -\sin(\theta)\cos(\phi) - \sin(\phi)\cos(\theta))\\
 \notag  &=&\frac{1}{h^2}(-2\sin(\theta)\cos(\phi)+2\sin(\theta))\\
  \notag  &=&\frac{1}{h^2}(2\sin(\theta)(1-\cos(\phi)))
\end{eqnarray}

And $1-\cos(\phi) = 2\sin^2(\frac{\phi}{2})$

\begin{eqnarray}
      (\mathbf{A}\mathbf{v}_j)_i 
\notag  &=&  \frac{4}{h^2}(\sin^2(\frac{\phi}{2})) \sin(\theta) =  \frac{4}{h^2}(\sin^2(\frac{j\pi h}{2})) \sin(j\pi i h)  =  \frac{4}{h^2}(\sin^2(\frac{j\pi h}{2})) \sin(j\pi x_i)\\
\notag &=& \frac{4}{h^2}(\sin^2(\frac{j\pi h}{2})) \mathbf{v_j}
\end{eqnarray}

This shows that $\mathbf{v}_j$ corresponds to an eigenvector of $\mathbf{A}$.

\textbf{2)} The corresponding eigenvalue $\lambda$ is the scalar multiplying the eigenvector.

\begin{equation}
    (\mathbf{A}\mathbf{v}_j)_i 
\notag   = \frac{4}{h^2}(\sin^2(\frac{j\pi h}{2})) \mathbf{v_j} = \lambda_j\mathbf{v_j}
\end{equation}

Then:

\begin{equation}
    \lambda_j= \frac{4}{h^2}\sin^2(\frac{j\pi h}{2}) = 4m^2\sin^2(\frac{j\pi}{2m})
\end{equation}

For $j = 1, \ldots, m-1$ (excluding the boundary nodes, where $\lambda$ is not defined in the same way).

\textbf{3)} If $m\rightarrow \infty$:
\begin{itemize}
	\item For $j = 1$, we obtain the smallest eigenvalue. Since 
	$\frac{\pi}{2m}$ is small, we use $\sin\left(\frac{\pi}{2m}\right) \approx \frac{\pi}{2m}$. Then:
	
	\begin{equation}
		\lambda_{\min} \approx 4m^2\left(\frac{\pi}{2m}\right)^2 = \pi^2
	\end{equation}
	
	\item For $j = m-1$, we obtain the largest eigenvalue. Since 
	\[
	\frac{(m-1)\pi}{2m} = \frac{\pi}{2} - \frac{\pi}{2m} \approx \frac{\pi}{2},
	\]
	we have $\sin\left(\frac{\pi}{2}\right) = 1$. Then:
	
	\begin{equation}
		\lambda_{\max} \approx 4m^2
	\end{equation}
\end{itemize}

Finally:

\begin{equation}
    \kappa(\mathbf{A}) \approx \frac{4m}{\pi} = O(m^2)
\end{equation}

\section{Problem 3}
\textbf{a)} 
\begin{eqnarray}
    u(x) = \sin(k\pi x) + c(x-\frac{1}{2})^3\\
    u'(x) = k\pi\cos(k\pi x ) + 3c(x-\frac{1}{2})^2\\
    u''(x) = -k^2\pi^2\sin(k\pi x) + 6c(x-\frac{1}{2})
\end{eqnarray}

Then, by susbtituting in $-u'' + \gamma u = f(x)$

\begin{eqnarray}
k^2\pi^2\sin(k\pi x) - 6c(x-\frac{1}{2}) + \gamma(\sin(k\pi x) + c(x-\frac{1}{2})^3) = f(x)\\
( k^2\pi^2+\gamma)\sin(k\pi x) + \gamma c(x-\frac{1}{2})^3 - 6c(x-\frac{1}{2}) = f(x)
\end{eqnarray}

\textbf{b)}

For this problem, the code was implemented in a file named bvp.c, located in the main HW2 directory. The file was developed based on the tri.c code and modified by adding the term $\gamma u$. As a result, the matrix $\mathbf{A}$ takes the following form:

\begin{equation}
	\mathbf{A}_i = \frac{1}{h^2}\begin{bmatrix}
-1 &2+\gamma &-1]
	\end{bmatrix}
\end{equation}

The right-hand side vector is given by $\mathbf{b} = f(x_i)$.

Using the provided plotting code and running the bvp.c program, the following plot is obtained:

\begin{figure}[H]
	\centering
	\includegraphics[width=12cm]{Figures/F1.png}
	\caption{Numerical vs exact solution}
	\label{fig:solution}
\end{figure}

\textbf{c)} By setting $\gamma = 0$ and $c = 0$, and using 1 MPI process, the following convergence plot is obtained:

\begin{figure}[H]
	\centering
	\includegraphics[width=12cm]{Figures/F2.png}
	\caption{Convergence plot}
	\label{fig:solution}
\end{figure}

It can be observed that the order of convergence is 2 for all values of $k$.


\section{Problem 4}



\begin{itemize}
	\item[(a)] Jacobi preconditioned Richardson (\textbf{10000 iterations}):
	Very slow convergence. The method essentially stagnates, since Richardson is highly sensitive to the condition number and Jacobi preconditioning is not sufficient to significantly improve convergence.
	
	\item[(b)] Unpreconditioned Conjugate Gradient (\textbf{102 iterations}):
	Significant improvement over Richardson. However, it still requires many iterations due to the condition number of the system and the absence of preconditioning.
	
	\item[(c)] Unpreconditioned Conjugate Gradient with $c = 0$ (\textbf{1 iteration}):
	Converges in a single iteration. This occurs because the right hand side aligns with an eigenvector of the matrix, allowing to recover the exact solution immediately.
	
	\item[(d)] ICC preconditioned Conjugate Gradient (\textbf{1 iteration}):
	Excellent performance. The ICC preconditioner provides a very accurate approximation of the matrix factorization, resulting in a well conditioned system.
	
	\item[(e)] Block Jacobi preconditioned CG (4 processors, \textbf{7 iterations}):
	Slower than ICC. The preconditioning is applied locally to each block rather than globally, which reduces its effectiveness due to loss of global coupling.
	
	\item[(f)] MUMPS direct solver (1 processor, \textbf{1 iteration}):
	Direct solver, so the solution is obtained exactly in one step. No iterative process is required.
	
	\item[(g)] MUMPS direct solver (4 processors, \textbf{1 iteration}):
	Same behavior as with one processor. Parallelization improves computational time, but not the number of iterations.
\end{itemize}

\end{document}
